\documentclass[10pt,letterpaper]{report}
\usepackage[utf8]{inputenc}
\usepackage{amsmath}
\usepackage{amsfonts}
\usepackage{amssymb}
\usepackage{graphicx}
\usepackage{subfig}
\author{Brandon Houghton}
\begin{document}
	
	
	In this period we explored the effect of various objective coefficients on training networks to predict orbital trajectories. We discovered potential candidate parameters, noting many initialization conditions found a non-trivial invariant with increasing explainability via a linear model of know parameters as training progressed. 
	
	Some variance remained unexplained by this method of approximating phi as a linear combination of known invariants, however, given the small coefficients in the linear model small errors in floating point accuracy can cause large shifts. In general across the sweep the ratio of explained variance showed good linear fit among reasonable parameters (those for which the gradients did not diverge). 
	
	Overall we have developed a framework that is capable of loading in data from physical sources and training a network that is capable of learning non-trivial functions of the dynamics of the system. These learned networks have shown to be intimately tied to known physical invariants. We have also demonstrated the ability for such networks to predict future trajectories enabling the network to embed the dynamics of a system within its activations. A network diagram is shown below in figure \ref{graph}. The tree input nodes are ellipses marked X, Xp, and F representing the state, previous state (used for predicting system dynamics if enabled), and the gradient of the current state respectively.
	
	
	
	\begin{figure}
		\begin{center}
			\includegraphics[width=0.7\textwidth]{images/ratio_RMS_to_var.PNG}
			\caption{\small RMS error between learned phi and best linear fit normalized by the variance of phi. The best sample achieved a ratio of  0.98 demonstrating nearly all of the variance was explained by known invariants. }
		\end{center}	
	\end{figure}
	
	
	
	
	\begin{figure}
		\begin{center}
			\includegraphics[width=0.99\textwidth]{images/graphArch.png}
			\caption{\small Illustration of network used to learn invariants in physical systems. A single dense layer is used for the base network and a fully connected layer joins the output of the base network with the single network output Phi, representing the learned invariant.}
			\label{graph}
		\end{center}	
	\end{figure}
	
	
\end{document}